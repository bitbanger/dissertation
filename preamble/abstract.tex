\begin{abstract}

To fully understand natural stories, a reader must infer many unstated facts, entities, and motivations.
Because these inferences require an immense amount of common sense knowledge about the world---an amount too large for researchers hand-craft \textit{a priori}---we must look to automatic knowledge acquisition techniques.
%%Broad language understanding calls for the development and integration of a rich knowledge representation, an accurate semantic parser, tractable inference procedures, and scalable methods of learning coherent world knowledge.%%
%LU does not stop with an accurate semantic parse of the surface text; it also relies on significant ``common sense'' via inference procedures and a broad corpus of world knowledge.
Modern machine learning approaches to knowledge acquisition require a huge amount of data, and yet generally fail to derive coherent knowledge representations or robust reasoning techniques.
Children, however, seem to learn highly general patterns of events not from thousands or millions of concrete examples, but often with only one or two.

This dissertation focuses on the definition and acquisition of knowledge in the form of ``schemas'': general patterns of events and facts that describe the core features of real-world situations. I introduce a rich and expressive schema representation based on a formal logic and show that, using a modest set of initial ``bootstrapping'' schemas and several state-of-the-art neural technologies, these schemas can be feasibly acquired from knowledge encoded in large language models. Consulting human judges, I also show that the acquired schemas are generally judged to be topically cohesive and sensible, and that the inferences generated from those schemas about unseen stories also tend to be judged as plausible.
  
\end{abstract}