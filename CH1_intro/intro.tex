Our knowledge of the world and its ways underlies our unmatched inferential capacity.
It is only by the hard-won acquisition of patterns of truths and behaviors in academic culture, for example, that I can infer that you are probably reading this document prior to its publication, in the role of a reviewer.
This kind of situational inference is the product of an analogical reasoning process by which we compare a situation at hand to a model of similar situations; assess properties of the latter that are consistent with, but unobserved in, the former; and tentatively assume the truth of those properties.
To continue my earlier example: I have never written a dissertation, and am aware of no explicit axiom prohibiting the kind of informal, discursive, example-oriented style of introduction I prefer to employ.
But I \textit{do} have experience with publishing other academic works, and in my model of \textit{that} kind of situation, I am not infrequently asked to ``button up'' any prose not in line with the (similarly hard-won) norms of academic writing; this situational analogy lets me infer, with some probability or another, a similar outcome in this situation.
Regardless of the introductory essay at its head, though, this dissertation concerns itself with the formal modeling, unsupervised acquisition from stories, and predictive capacity of these situational models, henceforth \textit{schemas}.