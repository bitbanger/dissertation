Two components are necessary for schema learning by definition: a model of schemas to learn, and an algorithm to learn them. In this chapter, we will examine several major approaches to schema representations and schema learning systems, noting in each case some limitations. There has been work on both classical, symbolic approaches to schema learning and modeling; and statistical, connectionist approaches (though far more of the latter, in contemporary literature). As we will see by examining several projects, work on the former tends to lack scalability due to demand for manual data construction, and work on the latter tends to limit the expressivity of the learned schema-like knowledge--often representing events as subject-predicate-object triples, or slight extensions thereof---to facilitate unsupervised computational learning.