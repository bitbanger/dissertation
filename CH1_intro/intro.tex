Our knowledge of the world and its ways underlies our unmatched inferential capacity.
It is only by the hard-won acquisition of patterns of truths and behaviors in academic culture, for example, that I can infer that you are probably reading this document prior to its publication, in the role of a reviewer; very few theses, after all, are said to be read after their completion.
This kind of situational inference is the product of an analogical reasoning process by which we compare a situation at hand---here, writing a dissertation---to a model of similar situations; assess properties of the latter that are consistent with, but unobserved in, the former; and tentatively assume the truth of those properties.
While simple ``if-then'' inference rules are often useful---e.g. \textit{if I see smoke, then there is probably a fire}---most situations in reality are complex and multi-faceted. Additional inferences may spring from an observation of smoke: perhaps a firefighter will arrive, and perhaps they will then put the inferred fire out. The predicted arrival and firefighting events, too, may erupt into other, perhaps more mundane, inferences---about the spatiotemporal trajectories of the entities involved, for example, or the relative sizes of people and firetrucks---which inferences are, regardless of their mundanity, frequently important in human-level reasoning. Furthermore, the entities mentioned in an inference, e.g. the fire, must share its identity with the fire entity mentioned in a co-occurring inference. A knowledge representation capable of encapsulating such rich networks of co-occurring facts and co-referring entities is called an \textit{event schema}.


%To continue my earlier example: I have never written a dissertation, and am aware of no explicit axiom prohibiting the kind of informal, discursive, example-oriented style of introduction I prefer to employ.
%But I \textit{do} have experience with publishing other academic works, and in my model of \textit{that} kind of situation, I am not infrequently asked to ``button up'' any prose not in line with the (similarly hard-won) norms of academic writing; this situational analogy lets me infer, with some probability or another, a similar outcome in this situation.
%Regardless of the introductory essay at its head, though, this dissertation concerns itself with the formal modeling, unsupervised acquisition from stories, and predictive capacity of these situational models, henceforth \textit{schemas}.