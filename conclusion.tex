\chapter{Conclusion}

This dissertation describes my work on EL schemas, a rich, symbolic representation of event knowledge. EL schemas model entities, their types, and relations between them, using variables that can be bound to specific individuals in observed stories. Once bound, these variables may be substituted throughout the schema, enabling inferences from partial information. EL schemas also model temporal relations between events, and allow events within a schema to represent nested schemas, making EL schemas hierarchical. EL schemas also model the goals, preconditions, and postconditions essential to understanding much human behavior and reasoning about causality.

EL schemas, despite their richness, are feasibly acquirable without continued manual construction. By ``bootstrapping'' with a neural FrameNet parser to identify specific instances of a small set of general \textit{protoschemas} in text, progressively more complex schemas can be built on the basis of already known schemas. The text in which protoschemas are identified may be any dataset of stories, but in creating NESL, the Neuro-Episodic Schema Learner, I showed that large, pre-trained language models can be induced to generate topically related datasets of stories for general purpose schema learning, and that the schemas learned for these stories can be algorithmically generalized together to abstract away details while retaining important descriptions of the sampled event types. NESL can be viewed either as a dataset augmenter serving a proof of concept of schema learning from stories, or as a demonstration that general situational knowledge can be extracted from language models and rendered into a symbolic form.

Finally, I showed that EL schemas learned by NESL, and inferences produced by them, are generally rated to be sensible. By verbalizing the learned schemas and asking human judges to rate the quality of the steps of the learned schemas given the topic, I found that NESL-learned schemas contain sensible, topically relevant steps at a good level of generality. Human judges also evaluated inferences produced by NESL-learned schemas about unseen stories, ultimately finding these inferences to be fairly plausible and relevant on average, especially in the case of entity type inferences.

Future work in this direction should include the development of a wider array of initial protoschemas, the integration of imagistic and physical knowledge into symbolic schemas, and general improvement of semantic parsing to cover more complex sentences, paraphrases, idioms, and other aspects of natural language that were not well represented by the simple stories examined here. Further work on the application of schemas is also underway, including the use of EL schemas for the representation of dialog \citep{lissa,kane2022flexible}; for the management of cooperative construction in a ``blocks world'' \citep{platonov2019spoken}; and for the representation of object knowledge, including the common uses and general shapes of objects, in a schematic form that can be integrated with event schemas, which work is ongoing in our research group.