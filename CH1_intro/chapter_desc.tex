\section{Chapter Descriptions}
Chapter~\ref{chap:related} discusses some notable related work on schema modeling and acquisition, including seminal work on the definition of schema-like knowledge (Sections~\ref{sec:schank}~and~\ref{sec:framenet}); the manual assemblage of corpora of such knowledge (Section~\ref{sec:framenet}); and its automatic acquisition from texts by both symbolic (Section~\ref{sec:ipp}) and statistical (Sections~\ref{sec:chambers}~and~\ref{sec:pichotta}) means. It concludes with a framing of my own dissertation work in the context of these prior efforts (Section~\ref{sec:related_disc}).

Chapter~\ref{chap:el} discusses \textbf{Episodic Logic} (EL), a formal semantic representation of language used to build our schema framework. Section~\ref{sec:ulf-parsing} covers the automatic parsing of an underspecified variant of EL, including novel experiments with a new parser for that variant, and Section~\ref{sec:parsing} discusses the conversion of that variant into the full EL form, whose various error types I decompose and analyze.

Chapter~\ref{chap:schemas} introduces \textbf{EL schemas}, the schematic knowledge representation around which this dissertation centers. It describes the form and semantics of EL schemas before outlining the process by which EL schemas are meant to be learned and the role of \textit{protoschemas} in that process. It shows that a set of fewer than 100 protoschemas can cover more than 80\% of actions in an example story corpus. Finally, it outlines and discusses a process by which stories are ``matched'' to schemas, and how inferences are generated by that matching process.

Chapter~\ref{chap:learning} describes \textbf{NESL}, the neuro-episodic schema learner, a collection of projects together forming the first step toward the automatic acquisition of EL schemas from text. Section~\ref{sec:lome} describes a neural protoschema identifier. Section~\ref{sec:lss} describes \textbf{latent schema sampling}, the procedure by which NESL learns schemas without a fixed training corpus. Section~\ref{sec:schemagen} discusses the generalization of multiple similar schemas. The chapter then concludes with a discussion of NESL's contributions within the wider scope of the EL schema learning project.

Chapter~\ref{chap:eval} covers two human-judged experiments performed to evaluate the schemas learned by NESL. Section~\ref{sec:verbviz} describes the verbalization and visualization of the EL formulas within schemas to make their meaning accessible to untrained human judges. Section~\ref{sec:schema_eval} then discusses the evaluation of the quality of schema formulas as general statements about situations, which produced an average step quality rating of 73.5\% across all learned schema steps, making the system competitive with contemporary schema learners. Section~\ref{sec:inf_eval} then discusses the evaluation of inferences generated by learned schemas about unseen stories, which produced fairly high ratings for the plausibility and relevance of the schema inferences. The chapter also theorizes about why inference ratings were generally slightly lower than those of schema steps not grounded in story context.