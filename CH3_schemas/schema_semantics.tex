\section{Schema Semantics}
Here, we define much of the semantics of EL schemas in terms of EL semantics. It is important to note, however, that the semantics of schema \textit{use} for inference are not necessarily well-defined in terms of EL semantics. As an example, consider the decision of whether the invocation a schema $\mathcal{S}$ is ``confirmed'' in an observed story. Some subset of the schema's variables $v \subset vars(\mathcal{S})$ have been bound to individuals in the story, and some subset of the schema's step episodes $e \subset steps(\mathcal{S})$ have been matched to steps in the story. To confirm the invocation of $\mathcal{S}$ and infer the presence of its unobserved formulas, we must map $v$ and $e$ to a truth value in $\{0, 1\}$. Such a function is schema-dependent and can be extremely complex; it likely demands an external heuristic to be practically evaluable, and as such is not explicitly modeled in EL schema semantics. So, while all of the semantics of what \textit{is} represented in an EL schema can easily be written using EL semantics, we must bear in mind that the semantics of the wider schema \textit{system}---which includes matching, inference, and learning---are far more complex, and thus not represented explicitly in the schemas themselves.

\subsection{Scoping and Skolemization}
The header episode of each schema $\mathcal{S}$ characterizes a header episode, which we will refer to, without loss of generality, as \el{?E}. An EL schema's variables are not explicitly quantified in any of its formulas; instead, their values are implicitly given by a Skolem function of \el{?E}. We refer to the domain of the Skolem function, which is equal to the set of free variables in the schema, as the schema's \textit{scope}. Each compositionally nested schema $\mathcal{S}_{i}$, represented as step $\texttt{?E}_{i} \in steps(\mathcal{S})$, has its own Skolem function and its own scope; scope is not shared with a compositional parent schema unless explicitly bound in the \texttt{:Subordinate-constraints} section, or implicitly bound when unifying the parent schema's step formula for $\texttt{?E}_{i}$ with the header formula of $\mathcal{S}_{i}$.

The Skolem function of schema $\mathcal{S}$ characterizing episode \texttt{?E}, which maps variable names to values, can be written as the symbol $\texttt{?E}\hspace{-2mm}\rightarrow$. Likewise, the Skolem functions of each nested schema $\mathcal{S}_{i}$ can be written as $\texttt{?E}_{i}\hspace{-2mm}\rightarrow$. This notation may be used in the \texttt{:Subordinate-constraints} section of each schema to explicitly bind a variable in a nested schema's scope to one in the parent schema's scope, as seen in Figure~\ref{fig:compo_hier}.

\subsection{Fluent and Non-Fluent Formulas}
Schemas organize their constituent formulas in two types of sections: \textit{fluent} and \textit{nonfluent} sections. Nonfluent sections, such as \texttt{:Roles} and \texttt{:Episode-relations}, whose formula IDs all begin with exclamation marks, contain formulas that hold true regardless of time. Fluent sections, such as \texttt{:Steps} and \texttt{:Preconditions}, contain formulas identified by variables starting with question marks; these formulas are susceptible to change over time. Formally, a fluent formula $\phi$ with episode ID \el{?E} is said to characterize its episode ID: ($\phi ** \texttt{?E}$). A non-fluent formula $\psi$, however, does not have an episode ID, as its truth value is not temporally constrained; instead, it is represented by a \textit{metavariable} \el{!E}, any occurrence of which in an EL formula may be equivalently \textit{substituted} with $\psi$.

\subsection{Temporal Relations}
Schemas characterize EL episodes that encompass their temporal duration. They also contain many other episodes, such as steps and preconditions, with their own temporal bounds. These episodes can all be complexly inter-related using constraints from the Allen Interval Algebra \citep{allen1983maintaining}. Pre- and post-conditions are implicitly constrained to be true at the start and end of the schema's header episode, respectively, and steps, by default, are ordered sequentially as listed in the schema, but additional constraints can be specified in the \textbf{Episode-relations} section of each schema. To evaluate these interval constraint propositions, we implemented a time graph specialist module \citep{gerevini1993efficient}.