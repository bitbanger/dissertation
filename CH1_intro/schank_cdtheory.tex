\iffalse
\subsection{Conceptual Dependency (CD) Theory}

Roger Schank's CD theory \citep{schank1975concdep} was an attempt to model sentence understanding in a way conducive to computation---that is, by creating a formal representation of sentences. Schank posited one essential ``axiom'' of CD theory: \textbf{any pair of sentences with the same meaning should have the same representation}. Although the sentences ``I sent her to the store'' and ``I sent our daughter to the grocery store'' may mean the same thing when uttered in a particular context, to represent the first sentence, all its implicit, contextual, and inferable information must be ``packaged up'' into the unique meaning representation of the utterance.

Clearly, we cannot package up all inferences drawable from the sentence and the context: there are potentially infinite such inferences, and many may not be interesting. So, Schank further posits a standard set of inference rules, and a representation conducive to inference---this is, after all, all being done with computation in mind. So, Schank settles on a representational scheme with the following components:

\begin{itemize}
    \item \textbf{Primitives}: ``atomic'' representations of actions in the world. For example, the verbs ``buy'' and ``sell'' should be decomposable into the same primitive actions. In fact, Schank posits that \textit{all} verbs can be decomposed into only about 11 primitive actions.
    \item \textbf{Active conceptualizations}: meaning propositions about actions undertaken by an actors, of the form \texttt{Actor Action Object Direction (Instrument)}.
    \item \textbf{Stative conceptualizations}: meaning propositions about the states of objects in the world, of the form \texttt{Object (is in) State (with Value)}.
\end{itemize}

The primitive actions in the CD theory meaning representation are described below, in Section~\ref{sec:activeconc}.

\subsubsection{Active conceptualizations}
\label{sec:activeconc}
\begin{itemize}
    \item \texttt{\textbf{PTRANS}}: changing the physical location of an object. So, ``Jake goes to school'' becomes \texttt{Jake \textbf{PTRANS} Jake \textbf{to} school}.
    
    \item \texttt{\textbf{ATRANS}}: transferring possession of something to another actor. \citep{schankandabelson} actually generalizes this a bit to ``an abstract relationship such as possession, ownership, or control''. A buying action and a selling action are both characterized by the same two \texttt{\textbf{ATRANS}} actions: one of money, and the other of an object.
    
    \item \texttt{\textbf{MTRANS}}: communicating mental information to another actor. They also define \texttt{\textbf{CP}}, the conscious processor, and \texttt{\textbf{LTM}}, long term memory, to be used in \texttt{\textbf{MTRANS}} propositions. For example, ``tell'' is a person-to-person \texttt{\textbf{MTRANS}}, and ``remember'' is an \texttt{\textbf{LTM}}-to-\texttt{\textbf{CP}} \texttt{\textbf{MTRANS}} (i.e., your memory is communicating information to your conscious mind, as if both were actors).
    
    \item \texttt{\textbf{MBUILD}}: synthesizing new information from known information. ``Deduce'', ``imagine'', and ``decide'' are examples of \texttt{\textbf{MBUILD}}-characterized actions.
    
    \item \texttt{\textbf{ATTEND}}: focusing a sense organ on an environmental stimulus. \texttt{\textbf{ATTEND} eye} is ``see'', \texttt{\textbf{ATTEND} nose} is ``smell'', etc. This is often used as the \texttt{Instrument} argument of active conceptualizations characterized by \texttt{\textbf{MTRANS}}: so ``see'' is, more accurately, \texttt{\textbf{MTRANS} to \textbf{CP} by instrument of \textbf{ATTEND} eye to object}.
    
    \item \texttt{\textbf{GRASP}}: grasping an object for manipulation. The verb ``throw'' can be expressed as a \texttt{\textbf{GRASP}} followed by the end of a \texttt{\textbf{GRASP}}.
    
    \item \texttt{\textbf{PROPEL}}: exerting physical force on something. That thing need not necessarily move as a result. ``John threw the ball'' can be expressed as the end of a \texttt{\textbf{GRASP}}, followed by a \texttt{\textbf{PROPEL}}.
    
    \item \texttt{\textbf{MOVE}}: moving one's own body part. \texttt{\textbf{MOVE}} is often used as an instrument for other actions: so ``John threw the ball'' will also be ``by instrument of '' John \texttt{\textbf{MOVE}}-ing his arm. \texttt{\textbf{MOVE}} is also the non-instrumental characterizing primitive for verbs like ``kiss'' or ``scratch''.
    
    \item \texttt{\textbf{INGEST}}: taking an object inside of one's self, e.g. ``eat'', ``drink'', and ``breathe''. The exact set of orifices \texttt{\textbf{INGEST}} supports is both unclear in the literature and probably best left unexplored.
    
    \item \texttt{\textbf{EXPEL}}: excretion, secretion, or otherwise ejection of an object from the body. ``Sweat'', ``exhale'', ``cry'', and many other colorful examples are characterized by \texttt{\textbf{EXPEL}}.
    
    \item \texttt{\textbf{SPEAK}}: producing a sound, not necessarily directly through one's own bodily apparatus, and, unlike \texttt{\textbf{MTRANS}}, not necessarily for the communication of a coherent piece of information. ``Squeal'', ``play music'', and ``purr'' all involve \texttt{\textbf{SPEAK}}.
\end{itemize}

The creative composition of this sort of primitive action to represent more complex actions was not a revolutionary idea to AI researchers at th time: as I detail in Section~\ref{sec:genesis}, the STRIPS system's \citep{fikes1972} ``macro-operations'' were hugely influential in the planning literature---no doubt including this book---due to the simplicity and power of the situation models and planning algorithms they facilitate.

\subsubsection{Stative conceptualizations}
Primitive actions form the basis for active conceptualizations (\texttt{\textbf{ACT}}s). Stative conceptualizations (\texttt{\textbf{STATE}}s), on the other hand, can be thought of as attribute-value pairs. Some states are qualitative, e.g. ``the sky is blue''. However, some are quantitative, and so Schank suggested a normalized \textit{scale}, from -10 to 10, for such attributes, such as \texttt{\textbf{HEALTH}} (-10: dead, diseased, under the weather, tolerable, 10: ``in the pink'' \footnote{I actually had to look this phrase up, which may date this research.}), \texttt{\textbf{AWARENESS}} (-10: dead, unconscious, asleep, awake, 10: keen) \footnote{-10 seems to be ``dead'' for an awful lot of the quantities Schank suggests.}, and more; Schank does not provide a definitive list of ``primitive quantities''.

\subsection{Causal chains}
In chapter 2 of the book, the authors introduce \textit{causal chains}, wherein a set of uniquely identifiable CD theory conceptualizations---both active and stative---are related. Put more simply: causal chains relate actions to the states they effect, the states that bring them about, and other actions.

They define the following set of causal relations:

\begin{itemize}
    \item $\Uparrow_{r}$: an \texttt{\textbf{ACT}} \textit{results in} a \texttt{\textbf{STATE}}.
    
    \item $\Uparrow_{E}$: a \texttt{\textbf{STATE}} \textit{enables} an \texttt{\textbf{ACT}}.
    
    \item $\Uparrow_{I}$: a \texttt{\textbf{STATE}} or \texttt{\textbf{ACT}} \textit{initiates} a mental \texttt{\textbf{STATE}}.
    
    \item $\Uparrow_{R}$: a mental \texttt{\textbf{ACT}} \textit{is the reason for} a physical \texttt{\textbf{ACT}}.
    
    \item $\Uparrow_{dE}$: a \texttt{\textbf{STATE}} \textit{disables} an \texttt{\textbf{ACT}}.
    
    \item $\Uparrow_{rE}$: an \texttt{\textbf{ACT}} leads to a \texttt{\textbf{STATE}}, which then enables an \texttt{\textbf{ACT}}. \textit{(This is an abbreviation for a $\Uparrow_{r}$-to-$\Uparrow_{E}$ chain.)}
    
    \item $\Uparrow_{IR}$: an \texttt{\textbf{ACT}} or \texttt{\textbf{STATE}} initiates some mental state, which is then the reason for an \texttt{\textbf{ACT}}. \textit{(This is an abbreviation for a $\Uparrow_{I}$-to-$\Uparrow_{R}$ chain.)}
\end{itemize}

So, ``John gave Bill an orange for his cold'' can be expressed as the causal chain \texttt{Bill \textbf{HEALTH}(POS \footnote{Here, \texttt{POS} indicates a positive change in quantity.} change) $\Rightarrow_{r}$ Bill \textbf{INGEST} orange \textbf{to INSIDE}(Bill) $\Rightarrow_{rE}$ John \textbf{ATRANS} orange \textbf{to} Bill} \footnote{Here, I've rotated the arrows 90\degree\ for formatting reasons.}.

\textit{N.B.: The relation arrows in the causal chain above seem to point in the ``wrong'' direction. I have preserved their direction in this example, which was originally given in \citep{schankandabelson}.}

Causal chains provide the first level of inter-sentential connection necessary to achieve situational understanding. But stories, and even observations in the world, frequently omit steps in causal chains. Some events also contain multiple causal chains, as they do not necessarily have one end goal: going to a restaurant for a date may have the goals of impressing a potential romantic partner \textit{and} eating some delicious food, but we nevertheless have certain stereotyped expectations about what may occur: one person may pull a chair out from the table for the other, one person will likely pay for all of the food, etc. So, we need an event representation richer than causal chains, not only to practically understand sparse stories, but also to model complex situation types.
\fi