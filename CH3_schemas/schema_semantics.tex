\section{EL Schemas and Formal Semantics}
\label{sec:formal_semantics}
As EL schemas are composed of EL formulas, it is tempting to attempt to specify their full semantics in the model-theoretic terms of EL semantics. It is important to note, however, that the semantics of schema \textit{use}---for matching, inference, and learning---are \textit{not} necessarily well-defined in terms of EL semantics. As an example, consider the decision of whether the invocation a schema $\mathcal{S}$ is ``confirmed'' in an observed story. Some subset of the schema's variables $v \subset vars(\mathcal{S})$ have been bound to individuals in the story, and some subset of the schema's step episodes $e \subset steps(\mathcal{S})$ have been matched to steps in the story. To confirm the invocation of $\mathcal{S}$ and infer the presence of its unobserved formulas, we must map $v$ and $e$ to a truth value in $\{0, 1\}$. Such a function is schema-dependent and can be extremely complex; it likely demands an external heuristic to be practically evaluable, and as such is not explicitly modeled in EL schema semantics. So, while the formulas in an EL schema are interpretable in EL semantics, we must bear in mind that the semantics of the wider schema \textit{system}---which includes matching, inference, and learning---are far more complex, and thus not represented explicitly in the schemas themselves.

Independently of its use or meaning in external systems and procedures, however, a schema can be viewed as a generic statement about the meaning of a kind of event. Much like a gloss uses a generic sentence to define the meaning of a word, an EL schema in its unbound form uses its fluent and non-fluent formulas to define the meaning of its header episode. For example, the schema in Figure~\ref{fig:libschema} purports to generically define what it means for an entity \el{?X}, of type \el{PERSON.N}, to \el{GO.V} to an entity \el{?L}, of type \el{LIBRARY.N}: such an event comprises two steps, \el{?E1} and \el{?E2}, which respectively represent the person's travel to the library and the person's subsequent search for a book. These episodes, the variables in their characterizing formulas, and the types of those variables describe all scenarios \el{?E} wherein a person goes to a library; the predication of their joint truth is a monadic predicate over all episodes, defining what it means to be an episode of a person going to a library. A forward inference rule, in which the predication of joint truth of all schema components implies the schema's header characterization (more or less certainly), and a reverse one, in which the schema's header characterization implies the certainty-weighted truth of all of its components, can be defined within the semantics of EL.

The inference from schema components to the header becomes weaker when based on partial observation, and the semantics becomes murkier. The failure of the example schema to fully describe our intuitive definitions of going-to-the-library events is also an example of the need for continual schema \textit{learning}, the semantics of which are also, as also mentioned above, difficult to render model-theoretically.
%Thus, the remainder of this section will discuss the EL semantics of schemas in this ``tableau'' interpretation, independently of matching, inference, and learning.