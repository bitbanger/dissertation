\begin{acknowledgments}

%\epigraph{\textit{Pretty much all I’ve ever done all the time is try to create a certain impression of me in other people. Mostly to be liked or admired.}}{--David Foster Wallace, \textit{Good Old Neon}}

Every key pressed and program executed in the pursuit of my doctorate has been due to someone else; if any action I took wasn't directly enabled by tireless support from my friends and family, then it was certainly at least motivated by my all-consuming drive to seem smart and impressive to everyone else. Here, I enumerate some of those from the former class, beginning, of course, with my advisor, Lenhart Schubert, whose patience, brilliance, good humor, and broad knowledge I as a researcher can only hope to crudely mimic. Thank you for putting up with me, Len.

My (readily overused) sense of humor and ability to put any situation in a new perspective, both inherited from my father, pervade my research and its presentation. The diligence and attention to detail I inherited from my mother have fuelled many late nights and joyful revelations. Thanks, too, to Scott, for all of the delicious home-cooked meals at mom's house, and to my grandmother for the chicken soup, grammatical competence (\textit{modulo} the occasional prepositionally concluded sentence), and ``topics of conversation'' she's gifted me with.

I am fortunate enough to have too many friends to feasibly name here. If you feel your name should be included but isn't, know that I thought of you specifically when choosing which names to include, and please write your name here: \underline{\hspace{50mm}}. My colleagues---e.g., Divya, Andrew, Komail, Gene, and Georgiy---made the office feel like home. Thanks are also due to my longtime friends and trivia teammates: Rachel, Jos, the Matts, Andrew (again), and ``this man'' have all made an underpaid and crushingly stressful time in a frigid hellscape much warmer. Thanks to Patsy for the phone calls about our doctoral experiences. Thanks, too, to my brother, Alex, who's always willing to strike up a conversation (or argument) about anything. And, finally, thank you to the women I've loved and who've loved me.

\end{acknowledgments}