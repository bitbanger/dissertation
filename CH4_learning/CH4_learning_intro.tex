This chapter describes a collection of projects comprising the \textit{Neuro-Episodic Schema Learner} (NESL). \footnote{Spoken aloud, ``NESL'' is pronounced like the English word \textit{nestle}.} In the previous chapter, I discussed the fundamental approach to EL schema acquisition: beginning with protoschemas, taxonomic specifications of known schemas are acquired via matching with text and composed into more complex schemas; these complex schemas are then stored and potentially generalized with other, similar schemas to focus on salient details at a reasonable level of generality. NESL is intended as an initial implementation of that general approach to schema learning; designed as a pipeline that incorporates pre-trained, state-of-the-art models, NESL leverages the flexibility of neural representation to match an initial set of protoschemas to text, and to learn coherent event schemas that exclude overly specific details.

In Section~\ref{sec:lome}, I describe NESL's use of a neural FrameNet parser to identify protoschemas in texts and form full EL protoschemas from those texts. Then, in Section~\ref{sec:nesl}, I describe the full NESL pipeline, including the \textit{latent schema sampling} (LSS) algorithm underlying the approach.
